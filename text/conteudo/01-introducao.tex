%!TeX root=../tese.tex
%("dica" para o editor de texto: este arquivo é parte de um documento maior)
% para saber mais: https://tex.stackexchange.com/q/78101/183146

%% ------------------------------------------------------------------------- %%
\chapter{Introduction}
\label{cap:introduction}

Social networks have all but taken over contemporary daily life. From the
eponymous socializing, to reading news, to expressing ourselves, social media
has creeped into every corner of society. Most of its side-effects, it could be
argued, are positive (shortening distances, political accountability, social
organizing), but they are not perfect institutions.

Social media companies already face significant backlash for their questionable
business model and ethics. Cambridge Analytica's election meddling, Facebook's
subliminal experiments, YouTube's problem with disturbing content marketed at
kids, and Twitter's bot infestation are just a few recent scandals that have put
the societal role of social media into question.

One particular controversy that has taken over public discourse around social
networks is the role that their algorithms might have in radicalizing users,
specially younger ones. The aforementioned experiments conducted by Facebook to
influence people's emotions and the proliferation of more than questionable
videos aimed at children on YouTube are instances that seem to corroborate the
notion that there is something fundamentally wrong with these companies'
algorithms.

News organizations, in general, have been skeptical of social networks.
Journalists and specialists alike argue that social media's algorithms
(specially recommender algorithms) are tuned to peddle conspiracy theories,
extremist views, and false information. This would be the source cause for a
plethora of what they consider contemporary evils: religious extremism,
anti-democratic leaders, widespread depression among teenagers, anti-science
movements, etc.

This narrative, of course, has been questioned for a variety of reasons. Some
say that it is self serving: traditional news organizations are being displaced
by social media and it would be convenient for them to mine the public's trust
in them. Others claim that these recommender algorithms are not to blame for
political polarization and that social networks even have a tendency to favor
more left-wing viewpoints.

The debate around the role of recommender systems in social media radicalization
is still, unfortunately, too recent and based in anecdotes. Since its impacts
are all but universal, more quality research is vital to inform both the public
and opinion makers about if and how much recommendation algorithms influence
social media users.

This dissertation aims to further such research. The rest of Chapter 1 is
dedicated to core concepts covered in the rest of the work, ending in subsection
1.4, which tackles the main hypothesis of this dissertation. Chapter 2 contains
the literature review, Chapter 3 explains the experiments already conducted and
their results, and Chapter 4 is about next steps.

\section{Social Networks}
\label{sec:social_networks}

Social networking services, also referred to as social networks and social
media, are notoriously difficult to define. Some definitions might be too narrow
(excluding instant messaging services), while some might be too broad (including
technologies such as telephone networks). Most definitions include some common
features:

\begin{itemize}
  \item Internet-based
  \item Focus on user-generated content
  \item Users have profiles
  \item Users can connect
\end{itemize}

While social-networking-like applications already existed in Usenet, Geocities,
launched in 1994, is usually regarded as the first major social network.
Friendster and Myspace followed in 2003, with Orkut and Facebook slightly
lagging behind in 2004. Each hit their peak at different moments and different
countries, but Facebook overtook all of them in 2009 when it became the most
popular social networking service in the world, still maintaining the title over
11 years latter at the moment of writing.

Even though all aforementioned social networks are multimedia, that is, users
can post text, photos and videos, some of the most popular services focus on a
specific type of media. For instance, YouTube (2009) centers around videos,
WhatsApp (2009) and WeChat (2011) were originally designed for text-based
communication, and Instagram's (2010) main focus is photos.

Some social media services, very much in agreement with McLuhan's teachings,
have what could be considered a ``style''. Instagram's content, for example,
tends toward more personal (i.e. egoic) photos and videos. As of November 28th,
2020, six of the top 20 most-liked posts on Instagram are from american
socialite Kylie Jenner, consisting of four photos of her daughter and two of her
ex-boyfriend. Even though there are many different niches inside Instagram,
personal posts seam to have an edge over other kinds of content.

Twitter, unlike most other social networks, allows for asymmetrical connections,
meaning users can follow profiles without being followed back. This enables the
emergence of Twitter communities (e.g. Fintwit, Black Twitter) that can be
largely self referential and/or organized around certain subjects. Facebook
users, on the other hand, can belong to groups, user-moderated profiles that
might revolve around any particular topics of interest; there are groups that
organize pet owners and groups that organize neonazis.

Parallel to all other features and idiosyncrasies, there lay the recommendation
algorithms. While a few social networking services (e.g. WhatsApp) do not
recommend any content or profiles to the user, most do and, according to recent
studies, these recommendations have become the main drivers of interactions.

\section{Recommender Systems}
\label{sec:recommender_systems}

Recommender systems (sometimes called recommendation systems or recomender
algorithms) first appeared in 1992 under the name ``collaborative filtering'',
even though that term nowadays refers to a subclass of recommender systems. The
aim of such an algorithm is providing users with personalized product or service
recommendations, an essential task when considering the ever increasing number
of possible videos to watch, music to listen, products to buy.

The input of a recommender system is usually information about the preferences
(ratings, likes/dislikes, watch time, etc.) of consumers for a set of items.
Preference information can be gathered from explicit behaviors (e.g. rating a
product in a scale ranging from 0 to 5 stars) or from implicit behaviors (e.g.
how much time the user lingers on a product's page). These data can be combined
with information about the user (age, political leaning, etc.) in order to
create the best possible representation of the user's preferences.

The output of these systems can come in the form of a prediction or a list of
recommended items. In the first case, the goal of the algorithm is approximating
the rating a user would attribute to a yet unrated item, while the second type
of output involves gathering the items that most likely would interest the user.
Simple recommender systems that suggest items similar to the one being queried
do not necessarily involve rating predictions, but it is common to have the list
of rcommended items based on the ratings the algorithms estimated the user would
give to those items.

Most recommender systems follow into one of four categories according to the
filtering algorithm they use, that it, the strategy for generating predictions
or selecting the top-N items: content-based filtering, demographic filtering,
collaborative filtering, and hybrid filtering.

Content-based filtering leverages characteristics of the content in order to
generate the recommendations. One such algorithm might use the genres of watched
movies in order to recommend new ones, while another might analyse the sound
signature of a song to recommend similar ones, but, either way, all
content-based systems establish a similarity between items as a basis for
recommendations. Analogously, demographic filtering uses demographic data to
establish a similarity between users and recommend items positively rated by
similar people.

Collaborative filtering algorithms also recommend items that similar users
liked, but, in this case, the similarity between users is based on past ratings
and not demographic information. Hybrid filtering usually mix collaborative
methods with either content-based or demographic filtering.

As with other knowledge-based systems, recommendation algorithms have quickly
incorporated neural networks and other machine learning techniques over the past
few years. Even though the implementation of YouTube's recommendation algorithm
is a trade secret, it is known to gather enormous amounts of data about the
user's interaction with the website and to require Google's own TPUs in order to
be trained. It also involves two distinct steps: candidate generation (when the
billions of videos available on the platform are quickly narrowed down to a few
hundreds that might be relevant) and ranking (when the algorithm actually
attempts to predict the score a user would implicitly give to the candidate
videos).

Another relevant aspect of recommender systems that is well-exemplified by
YouTube is the use of balancing factors such as novelty, dispersity, and
stability. In the case of Google's video giant, there is a baked-in bias for
recency, strongly favoring newer videos in detriment of older content.

\section{Radicalization}
\label{sec:radicalization}

Opinion polarization is far from a recent phenomenon, and social media is only
the most recent communication medium where it can be detected and studied. An
important question is whether it facilitates or attenuates polarization:
anecdotal evidence might suggest that social network structures incentivize
users to gather into antagonistic communities, but this could be a result of
people simply being more likely to express their preferences online, not of some
intrinsic property of social media.

One possible byproduct of polarization is radicalization. Despite not being
entirely different phenomena, these concepts deserve distinct levels of
attention. While polarization can be considered a natural part of democratic
discourse, radicalization only happens when certain conditions are met. UNESCO
defines radicalization as:

\begin{itemize}
  \item The individual person's search for fundamental meaning, origin and
        return to a root ideology;
  \item The individual as part of a group's adoption of a violent form of
        expansion of root ideologies and related oppositionist objectives;
  \item The polarization of the social space and the collective construction of
        a threatened ideal 'us' against 'them,' where the others are dehumanized
        by a process of scapegoating.
\end{itemize}

\section{Hypothesis}
\label{sec:hypothesis}

Escrever bem é uma arte que exige muita técnica e dedicação e,
consequentemente, há vários bons livros sobre como escrever uma boa
dissertação ou tese. Um dos trabalhos pioneiros e mais conhecidos nesse
sentido é o livro de
%Umberto Eco~\cite{eco:09} % usando o estilo alpha
Umberto~\citet{eco:09} % usando o estilo plainnat
intitulado \emph{Como se faz uma tese}; é uma leitura bem interessante mas,
como foi escrito em 1977 e é voltado para trabalhos de graduação na Itália,
não se aplica tanto a nós.

Sobre a escrita acadêmica em geral, John Carlis disponibilizou um texto curto
e interessante~\citep{carlis:09} em que advoga a preparação de um único
rascunho da tese antes da versão final. Mais importante que isso, no
entanto, são os vários \textit{insights} dele sobre a escrita acadêmica.
Dois outros bons livros sobre o tema são \emph{The Craft of Research}~\citep{craftresearch}
e \emph{The Dissertation Journey}~\citep{dissertjourney}. Além disso, a USP
tem uma compilação de normas relativas à produção de documentos
acadêmicos~\citep{usp:guidelines} que pode ser utilizada como referência.

Para a escrita de textos especificamente sobre Ciência da Computação, o
livro de Justin Zobel, \emph{Writing for Computer Science}~\citep{zobel:04}
é uma leitura obrigatória. O livro \emph{Metodologia de Pesquisa para
Ciência da Computação} de
%Raul Sidnei Wazlawick~\cite{waz:09} % usando o estilo alpha
Raul Sidnei~\citet{waz:09} % usando o estilo plainnat
também merece uma boa lida. Já para a área de Matemática, dois livros
recomendados são o de Nicholas Higham, \emph{Handbook of Writing for
Mathematical Sciences}~\citep{Higham:98} e o do criador do \TeX{}, Donald
Knuth, juntamente com Tracy Larrabee e Paul Roberts, \emph{Mathematical
Writing}~\citep{Knuth:96}.

Apresentar os resultados de forma simples, clara e completa é uma tarefa que
requer inspiração. Nesse sentido, o livro de
%Edward Tufte~\cite{tufte01:visualDisplay}, % usando o estilo alpha
Edward~\citet{tufte01:visualDisplay}, % usando o estilo plainnat
\emph{The Visual Display of Quantitative Information}, serve de ajuda na
criação de figuras que permitam entender e interpretar dados/resultados de forma
eficiente.

Além desse material, também vale muito a pena a leitura do trabalho de
%Uri Alon \cite{alon09:how}, % usando o estilo alpha
Uri \citet{alon09:how}, % usando o estilo plainnat
no qual apresenta-se uma reflexão sobre a utilização da Lei de Pareto para
tentar definir/escolher problemas para as diferentes fases da vida acadêmica.
A direção dos novos passos para a continuidade da vida acadêmica deveria ser
discutida com seu orientador.

%% ------------------------------------------------------------------------- %%
\section{Considerações de Estilo}
\label{sec:consideracoes_preliminares}

Normalmente, as citações não devem fazer parte da estrutura sintática da
frase\footnote{E não se deve abusar das notas de rodapé.\index{Notas de rodapé}}.
No entanto, usando referências em algum estilo autor-data (como o estilo
plainnat do \LaTeX{}), é comum que o nome do autor faça parte da frase. Nesses
casos, pode valer a pena mudar o formato da citação para não repetir o nome do
autor; no \LaTeX{}, isso pode ser feito usando os comandos
\textsf{\textbackslash{}citet}, \textsf{\textbackslash{}citep},
\textsf{\textbackslash{}citeyear} etc. documentados no pacote
natbib \citep{natbib}\index{natbib} (esses comandos são compatíveis com biblatex
usando a opção \textsf{natbib=true}, ativada por padrão neste modelo). Em geral,
portanto, as citações devem seguir estes exemplos:

\footnotesize
\begin{verbatim}
Modos de citação:
indesejável: [AF83] introduziu o algoritmo ótimo.
indesejável: (Andrew e Foster, 1983) introduziram o algoritmo ótimo.
certo: Andrew e Foster introduziram o algoritmo ótimo [AF83].
certo: Andrew e Foster introduziram o algoritmo ótimo (Andrew e Foster, 1983).
certo (\citet ou \citeyear): Andrew e Foster (1983) introduziram o algoritmo ótimo.
\end{verbatim}
\normalsize

O uso desnecessário de termos em língua estrangeira deve ser evitado. No entanto,
quando isso for necessário, os termos devem aparecer \textit{em itálico}.
\index{Língua estrangeira}
% index permite acrescentar um item no indice remissivo

Uma prática recomendável na escrita de textos é descrever as
legendas\index{Legendas} das figuras e tabelas em forma auto-contida: as
legendas devem ser razoavelmente completas, de modo que o leitor possa entender
a figura sem ler o texto onde a figura ou tabela é citada.\index{Floats}

Sugerimos que você faça referências bibliográficas de forma similar aos
estilos ``alpha'' (referências alfanuméricas) ou ``plainnat'' (referências
por autor-data) de \LaTeX{}.  Se estiver usando natbib+bibtex\index{natbib}\index{bibtex},
use os arquivos .bst ``alpha-ime.bst'' ou ``plainnat-ime.bst'', que são
versões desses dois formatos traduzidas para o português. Se estiver usando
biblatex\index{biblatex} (recomendado), escolha o estilo ``alphabetic''
(que é um dos estilos padrão do biblatex) ou ``plainnat-ime''. O arquivo de
exemplo inclui todas essas opções; basta des-comentar as linhas
correspondentes e, se necessário, modificar o arquivo Makefile para chamar
o bibtex\index{bibtex} ao invés do biber\index{biber} (este último é usado
em conjunto com o biblatex).

\section{Ferramentas Bibliográficas}

Embora seja possível pesquisar por material acadêmico na Internet usando sistemas
de busca ``comuns'', existem ferramentas dedicadas, como o \textsf{Google Scholar}\index{Google Scholar}
(\url{scholar.google.com}). Você também pode querer usar o \textsf{Web of Science}\index{Web of Science}
(\url{webofscience.com}) e o \textsf{Scopus}\index{Scopus} (\url{scopus.com}), que oferecem
recursos sofisticados e limitam a busca a periódicos com boa reputação acadêmica.
Essas duas plataformas não são gratuitas, mas os alunos da USP têm acesso a elas
através da instituição. Ambas são capazes de exportar os dados para o formato .bib,
usado pelo \LaTeX{}. Algumas editoras, como a ACM e a IEEE, também têm sistemas de
busca bibliográfica.

Apenas uma parte dos artigos acadêmicos de interesse está disponível livremente
na Internet; os demais são restritos a assinantes. A CAPES assina um grande
volume de publicações e disponibiliza o acesso a elas para diversas universidades
brasileiras, entre elas a USP, através do seu portal de periódicos
(\url{periodicos.capes.gov.br}). Existe uma extensão para os navegadores
Chrome e Firefox (\url{www.infis.ufu.br/capes-periodicos}) que facilita o uso
cotidiano do portal.

Para manter um banco de dados organizado sobre artigos e outras fontes bibliográficas
relevantes para sua pesquisa, é altamente recomendável que você use uma ferramenta
como Zotero~(\url{zotero.org})\index{Zotero} ou
Mendeley~(\url{mendeley.com})\index{Mendeley}. Ambas podem exportar seus dados no
formato .bib, compatível com \LaTeX{}. Também existem três plataformas
gratuitas que permitem a busca de referências acadêmicas já no formato .bib:

\begin{itemize}
  \item \emph{CiteULike}\index{CiteULike} (patrocinados por Springer): \url{www.citeulike.org}
  \item Coleção de bibliografia em Ciência da Computação: \url{liinwww.ira.uka.de/bibliography}
  \item Google acadêmico\index{Google Scholar} (habilitar bibtex nas preferências): \url{scholar.google.com}
\end{itemize}

Lamentavelmente, ainda não existe um mecanismo de verificação ou validação das
informações nessas plataformas. Portanto, é fortemente sugerido validar todas
as informações de tal forma que as entradas bib estejam corretas.

De qualquer modo, tome muito cuidado na padronização das referências
bibliográficas: ou considere TODOS os nomes dos autores por extenso, ou TODOS
os nomes dos autores abreviados.  Evite misturas inapropriadas.

\section{O Que o IME Espera}

Ao terminar sua tese/dissertação, você deve entregar uma cópia dela para a
CPG. Após a defesa, você tem 30 dias para revisar o texto e incorporar as
sugestões da banca. Assim, há duas versões oficiais do documento: a versão
original e a versão corrigida, o que deve ser indicado na folha de rosto.
\index{Tese/Dissertação!versões}

Fica a critério do aluno definir aspectos como o tamanho de fonte, margens,
espaçamento, estilo de referências, cabeçalho, etc. considerando sempre o
bom senso. A CPG, em reunião realizada em junho de 2007, aprovou que as
teses/dissertações deverão seguir o formato padrão por ela
definido\footnote{\url{www.ime.usp.br/dcc/pos/normas/tesesedissertacoes}}.
Esse padrão refere-se aos itens que devem estar presentes nas teses/dissertações
(e.g. capa, formato de rosto, sumário, etc.), e não à formatação do documento.
Ele define itens obrigatórios e opcionais, conforme segue:\index{Formatação}
\index{Tese/Dissertação!itens obrigatórios}
\index{Tese/Dissertação!itens opcionais}

\begin{itemize}
  \item \textsc{Capa} (obrigatória)
  \begin{itemize}
    \item O IME usa uma capa padrão de cartolina para todas as
    teses/dissertações.  Essa capa tem uma janela recortada por onde se
    vê o título e o autor do trabalho e, portanto, a capa impressa do
    trabalho deve incluir o título e o autor na posição correspondente da
    página. Ela fica centralizada na página, tem 100mm de largura, 60mm de
    altura e começa 47mm abaixo do topo da página.

    \item O título da tese/dissertação deverá começar com letra maiúscula
    e o resto deverá ser em minúsculas, salvo nomes próprios.

    \item O nome do aluno(a) deverá ser completo e sem abreviaturas.

    \item É preciso explicitar se é uma tese ou dissertação (para
    obtenção do título de doutor, tese; para obtenção do título de
    mestre, dissertação).

    \item O nome do programa deve constar da capa (Matemática,
    Matemática Aplicada, Estatística ou Ciência da Computação).

    \item Também devem constar o nome completo do orientador e do
    co-orientador, se houver.

    \item Se o aluno recebeu bolsa, deve-se indicar a(s) agência(s).

    \item É preciso informar o mês e ano do depósito ou da entrega da
    versão corrigida.
  \end{itemize}

  \item \textsc{Folha de Rosto} (obrigatória, tanto para a versão
  depositada quanto para a versão corrigida)
  \begin{itemize}
    \item o título da tese/dissertação deverá seguir o padrão da capa

    \item deve informar se se trata da versão original ou da versão
    corrigida; no segundo caso, deve também incluir os nomes
    dos membros da banca.
  \end{itemize}

  \item \textsc{Agradecimentos} (opcional)

  \item \textsc{Resumo}, em português (obrigatório)

  \item \textsc{Abstract}, em inglês (obrigatório)

  \item \textsc{Sumário} (obrigatório)

  \item \textsc{Listas} (opcionais)
  \begin{itemize}
    \item Lista de Abreviaturas
    \item Lista de Símbolos
    \item Lista de Figuras
    \item Lista de Tabelas
  \end{itemize}

  \item \textsc{Referências Bibliográficas} (obrigatório)

  \item \textsc{Índice Remissivo} (opcional\footnote{O índice remissivo
   pode ser muito útil para a banca; assim, embora seja um item opcional,
   recomendamos que você o crie.})
\end{itemize}
