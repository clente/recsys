%!TeX root=../tese.tex
%("dica" para o editor de texto: este arquivo é parte de um documento maior)
% para saber mais: https://tex.stackexchange.com/q/78101/183146

\chapter{Literature review}

Topic-Specific YouTube Crawling to Detect Online Radicalization
\citet{agarwal_topic-specific_2015}

Technologically scaffolded atypical cognition: the case of YouTube's recommender
system \citet{alfano_technologically_2020}

Recommender systems survey \citet{bobadilla_recommender_2013}

Evaluating the dynamic properties of recommendation algorithms
\citet{burke_evaluating_2010}

Do Search Algorithms Endanger Democracy? An Experimental Investigation of
Algorithm Effects on Political Polarization \citet{cho_search_2020}

Deep Neural Networks for YouTube Recommendations \citet{covington_deep_2016}

A Longitudinal Analysis of YouTube's Promotion of Conspiracy Videos
\citet{faddoul_longitudinal_2020}

The Statistical Properties of Random Bitstreams and the Sampling Distribution of
Cosine Similarity \citet{giller_statistical_2012}: this paper identifies certain
aspects of cosine similarity that are often overlooked. Starting from simple
theorems regarding the density of n-dimensional spheres, the authors conclude
that the expected cosine similarity between random bitstreams might be
significantly different from the average. This is noteworthy because many
recommendation algorithms use cosine similarity in order to determine the
similarity between two items to recommend.

Social media recommendation based on people and tags \citet{guy_social_2010}

Interactive recommender systems: A survey of the state of the art and future
research challenges and opportunities \citet{he_interactive_2016}

Evaluating the scale, growth, and origins of right-wing echo chambers on YouTube
\citet{hosseinmardi_evaluating_2020}

Diversity in recommender systems – A survey \citet{kunaver_diversity_2017}

Algorithmic Extremism: Examining YouTube's Rabbit Hole of Radicalization
\citet{ledwich_algorithmic_2019}

Right-Wing YouTube: A Supply and Demand Perspective
\citet{munger_right-wing_2020}: controversial article that postulates a new
model for YouTube radicalization. According to the authors, YouTube's algorithm
is not to blame, the users themselves are looking for extreme content and the
recommender system only supplies them. Its methods were highly questioned by the
community and is currently the only paper that spouses the supply and demand
hypothesis.

Analyzing Right-wing YouTube Channels: Hate, Violence and Discrimination
\citet{ottoni_analyzing_2018}

A study of the dynamic features of recommender systems \citet{rana_study_2012}

Auditing radicalization pathways on YouTube \citet{ribeiro_auditing_2020}: this
is one of the seminal articles that explore the radicalization pipeline
hypothesis of algorithmic enabled radicalization. The authors collect huge
amounts of YouTube comment data over time, and determine a significant migration
of users from ``lighter'' content towards more extreme videos. This doesn't
prove that the pipeline exists, but is a strong argument for its existence.

Algorithmic bias amplifies opinion fragmentation and polarization: A bounded
confidence model \citet{sirbu_algorithmic_2019}

Algorithmic Fairness for Networked Algorithms \citet{stoica_algorithmic_2020}

Hegemony in Social Media and the effect of recommendations
\citet{stoica_hegemony_2019}

Algorithmic Glass Ceiling in Social Networks: The effects of social
recommendations on network diversity \citet{stoica_algorithmic_2018}: this paper
explores the existence of an ``algorithmic glass ceiling'' and introduces the
concept of differentiated homophily. The authors experiment on a Instagram
dataset before and after the introduction of algorithmic recommendations and
discover that, even though most of that network's users were female, the most
followed profiles were male. They explain this phenomenon by postulating that
the algorithm learns biases in the population, that is, male preference for male
profiles (which doesn't happens for females and thus characterizes an
asymmetric---differentiated---homophily), and ends up enhancing this effect.

The Effect of Recommendations on Network Structure \citet{su_effect_2016}

An automated pipeline for the discovery of conspiracy and conspiracy theory
narrative frameworks: Bridgegate, Pizzagate and storytelling on the web
\citet{tangherlini_automated_2020}

Recommending what video to watch next: a multitask ranking system
\citet{zhao_recommending_2019}
