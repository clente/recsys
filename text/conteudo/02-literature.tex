%!TeX root=../tese.tex
%("dica" para o editor de texto: este arquivo é parte de um documento maior)
% para saber mais: https://tex.stackexchange.com/q/78101/183146

\chapter{Literature review}
\label{cap:review}

There are three types of work that are relevant to the current topic: general
literature about recommender systems, evidences of algorithmic bias, and methods
of creating fairer recommendations. Since this area of study is still mostly
unexplored, there is no consensus on whether social media recommender systems
favor extremist content (or even whether they are actually deradicalisation
agents), which means that many references used in this work might disagree
amongst themselves.

\section{Scientific literature}
\label{cap:scientific}

General literature about recommendation algorithms is abound. One of the most
cited surveys was elaborated by \citet{bobadilla_recommender_2013}, but works by
\citet{he_interactive_2016} (about interactive recommender systems), by
\citep{nguyen_exploring_2014} (about filter bubbles), and by
\citet{kunaver_diversity_2017} (about diversity in recommender systems) were
also used in order to draw a complete panorama of the field.

Another relevant article, by \citet{guy_social_2010}, is the landmark paper that
inaugurates the usage of user data alongside labels to create a recommendation
algorithm that is highly accurate and a staple of modern social networks. This
essentially starts the usage of recommenders systems in social media.

When talking specifically about YouTube's recommendation algorithms, two papers
deserve special attention. The first one, by \citet{covington_deep_2016}, marks
YouTube's move towards the usage of deep neural networks to generate video
recommendations. The authors describe a two-stage model that first generates a
list of candidates and then ranks them, also reporting dramatic performance
improvements. The second one, by \citet{zhao_recommending_2019}, describing a
more recent version of YouTube's recommendation algorithm, explores the
Multi-gate Mixture-of-Experts technique to optimize recommendations for more
than one ranking objective and the Wide \& Deep framework to mitigate selection
biases. The authors also make it clear that YouTube's recommender system has a
strong bias towards more recent content instead of more traditional metrics.

Many authors have also explored how biases in recommendation engines might lead
to user radicalization. \citet{agarwal_topic-specific_2015} developed an early
example of a technique to try and find extremist content on YouTube. Using
advanced machine learning methods, the authors create a YouTube crawler that
starts from a seed video and iteratively classifies featured channels and videos
according to their potential extremism. A more recent example of this can be
found in \citet{tangherlini_automated_2020}, where the authors propose a novel
approach for identifying conspiracy theories online. By analyzing the narrative
structure of a conspiracy theory (Pizzagate) and comparing it to an actual
conspiracy (Bridgegate), they create a model that can guess whether a
conspiratorial narrative is or not fabricated. According to their findings, a
multi-domain nature and the presence of keystone nodes are signs that strongly
indicate a conspiracy theory.

Besides just finding and identifying radicalizing content on YouTube, many
authors have been concerned with studying the radicalization dynamics directly.
\citet{alfano_technologically_2020}, for example, claim to be ``the first
systematic, pre-registered attempt to establish whether and to what extent the
recommender system tends to promote such [extremist] content.''
\citet{cho_search_2020} also attempt to understand how users can be radicalized
by the algorithm. By experimentally manipulating user search/watch history, the
authors concluded that algorithmically recommended content can reinforce a
participant's political opinions.

In the same vein, \citet{faddoul_longitudinal_2020}, after some high-profile
cases of users being radicalized through YouTube videos, studied the efforts
announced by the platform to curb the spread of conspiracy theories on the
website. The paper aimed to verify this claim by developing both an emulation of
YouTube's recommendation algorithm and a classifier that labeled whether a video
is conspiratorial or not. The authors describe an overall decrease in the number
of conspiracy recommendations, though not when weighing these recommendations by
views.

Three papers that deserve a closer look are those that investigate how regular
recommendation algorithms can learn covert biases in the users of a social
network and amplify them to previously unimaginable rates.
\citet{stoica_algorithmic_2018} explore the existence of an ``algorithmic glass
ceiling'' and introduces the concept of differentiated homophily. The authors
experiment on a Instagram dataset before and after the introduction of
algorithmic recommendations and discover that, even though most of that
network's users were female, the most followed profiles were male. They explain
this phenomenon by postulating that the algorithm learns biases in the
population, that is, male preference for male profiles (which doesn't happens
for females and thus characterizes an asymmetric---differentiated---homophily),
and ends up enhancing this effect. \citet{stoica_hegemony_2019}, building on top
of their previous work, create a proposal for new recommender systems that take
differentiated homophily into account in order to reduce the ``glass ceiling''
effect observed in non-corrected recommendation algorithms. The work focuses on
the theoretical description of the algorithm, but also attempts to validate its
hypothesis in real world data. \citet{stoica_algorithmic_2020}, in their most
recent paper, show that the most commonly used metrics in recommender systems
``exacerbate disparity between different communities'' because they reinforce
homophilic behavior of the network. This has profound implications, since these
algorithms might further suppress already minoritary viewpoints without being
explicitly programmed to do so.

Like the aforementioned articles, \citet{matakos_maximizing_2020} also propose a
novel recommendation algorithm that tries to strike a balance between
information spread and ensuring that the users are exposed to diverse
viewpoints. The authors show that this goal is important if we want to foster
healthy online debate, and that the algorithm is efficient and scalable with a
minor approximation. One possible inspiration for these papers might be one by
\citet{su_effect_2016} that studied the network structure of Twitter before and
after the introduction of algorithmic recommendations (``Who to Follow''). The
authors of the paper discovered that all users benefitted recommendations, but
that users with already popular profiles benefitted even more, effectively
changing the network structure and dynamics. \citet{caton_fairness_2020} have
recently compiled other valuable information on fairness in machine learning
into a survey.

Because of data limitations, there still are few studies that investigate how
recommendation algorithms work dynamically, over time.
\citet{burke_evaluating_2010} point out that most methods for evaluating
recommender systems are static, that is, involve static snapshots of user and
item data. The authors propose a novel evaluation technique that helps provide
insight into the evolution of recommendation behavior: the ``temporal
leave-one-out'' approach. A more recent example of this approach was developed
by \citet{roth_tubes_2020}. Their paper delves into the confinement dynamics
possibly fostered by YouTube's recommendation algorithm. The authors create,
from a diverse set of seed videos, a graph of the videos iteratively recommended
by YouTube and, from this, study whether there were created ``filter bubbles''.
They find that indeed YouTubes recommendations are prone to confinement dynamics
be it topological, topical or temporal.

Even more recently, \citet{yao_measuring_2021} propose an approach for measuring
recommender system bias based on simulated users. Even though this work focuses
only on bias towards popular content, it is of particular importance because it
was written by researchers from Google itself. Some years before,
\citet{dash_network-centric_2019} also proposed a framework for auditing
recommender systems based on its network of users. Another contribution of their
work is a novel quantification of diversity.

A different approach to understanding biases in recommendation algorithms range
from analyzing similarity metrics to developing theoretical bounded confidence
models. \citet{giller_statistical_2012} goes with the first strategy, and
identifies certain aspects of cosine similarity that are often overlooked.
Starting from simple theorems regarding the density of n-dimensional spheres,
the author concludes that the expected cosine similarity between random
bitstreams might be significantly different from the average. This is noteworthy
because many recommendation algorithms use cosine similarity in order to
determine the similarity between two items to recommend.
\citet{sirbu_algorithmic_2019} go with the latter, providing an interesting
theoretical model of how inherent biases in algorithmic recommendations might
heighten opinion polarization. Using a bounded confidence model, the authors
propose the addition of a term that represents the odds of an algorithm
recommending content that differs from that of a user.

Some recent papers also try to understand how YouTube might be favoring
right-wing and fascist content in specific, as opposed to trying to prove a more
general (and possibly less tractable) claim.
\citet{hosseinmardi_evaluating_2020} find evidence via a longitudinal study that
there exists ``a small but growing echo chamber of far-right content
consumption'' on YouTube. According to their research, these users are more
engaged than most, with YouTube generally accounting for a larger share of their
online news diet than the average. The authors, however, find no evidence of
this phenomenon being due to recommendations. A popular article in the field, by
\citet{ribeiro_auditing_2020}, explored the radicalization pipeline hypothesis
of algorithmic enabled radicalization. The authors collect huge amounts of
YouTube comment data over time, and determine a significant migration of users
from ``lighter'' content towards more extreme videos. This does not prove that
the pipeline exists, but is a strong argument for its existence.

Twitter was also found to consistently favor right-wing content.
\citet{huszar_algorithmic_2021} conducted a ''long-running, massive-scale
randomized experiment`` across 7 countries in order investigate the effects of
algorithmic personalization on users' feeds and, according to their results,
``mainstream political right enjoys higher algorithmic amplification than the
mainstream political left''.

Finally, feedback loops are of special interest to this discussion. Caused by
the inevitable fact that recommender systems must learn from users' reactions to
its own recommendations, they are widely believed to be a powerful engine of
bias amplification and are discussed at length in the literature. Already in the
last decade, \citet{sinha_deconvolving_2017} investigated the viability of
identifying items affected by these feedback loops and attempted to created a
method of deconvolving them. More recently, \citet{jiang_degenerate_2019}
explored what they called ``degenerate feedback loops'' and their capability of
creating echo chambers, going as far as proposing a novel approach of slowing
down this tendency towards degeneracy. In a related study,
\citet{mansoury_feedback_2020} explored how recommender systems amplify already
popular content, but, more importantly, how this tendency might reduce content
diversity and cause users' tastes to shift over time. Depending on what a
systems values (recency, virality, controversy, engagement), this type of
feedback loop could possibly amplify not ``popular'' content, but divisive and
extremist content.

The hypothesis that social networks have a radicalizing tendency is far from the
only one supported by research; some argue that radicalization happens through
different mechanisms or that it happens to only one subset of users.
\citet{munger_right-wing_2020}, for example, published an article that
postulates a new model for YouTube radicalization. According to the authors,
YouTube's algorithm is not to blame, the users themselves are looking for
extreme content and the recommender system only supplies them; this supply and
demand hypothesis was questioned by the scientific community, but its results
are nonetheless significant. \citet{ledwich_algorithmic_2019} also wrote a
controversial paper where its authors claim to have found evidence to support
the hypothesis that YouTube's recommendation algorithm favors mainstream and
left-leaning channels instead of right-wing ones. They categorize almost 800
channels into groups of similar political leaning and analyze recommendations
between each group, finding that YouTube might actually discourage users from
viewing radicalizing content. In an even earlier study on news recommendations
of a major Dutch newspaper, \citet{moller_not_2018} claim that recommenders
systems had no significant impact on content diversity.

Even with a quickly growing body of research, further studies are needed in
order to shed more light into the inner workings of how recommendation
algorithms are used by social networks. Articles like the ones described in this
chapter are of utter importance to this task, but generalist studies that are
able to capture dynamics common to all or most recommender systems are still
nonexistent.

\section{Journalistic efforts}
\label{cap:journalistic}

Since this field of study is still in its infancy, many relevant sources are not
scientific in nature. Journalism, specially when investigative in nature, is a
valuable ally when trying to understand what is happening behind the curtains of
social platforms.

Some examples of journalistic endeavors that inform and guide scientific
research include, but are not limited to, a series by \citet{lecher_one_nodate}
on how different are Americans' Facebook feeds, a report (in Portuguese) by
\citet{ribeiro_como_2021} on how the far-right is still able to cheat YouTube's
attempts at curbing extremist content, and a whistleblower's account to
\citet{wong_how_2021} of how Facebook's executives resist on restricting
fake engagement that is able to distort global politics.
