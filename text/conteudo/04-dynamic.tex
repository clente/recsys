%!TeX root=../tese.tex
%("dica" para o editor de texto: este arquivo é parte de um documento maior)
% para saber mais: https://tex.stackexchange.com/q/78101/183146

\chapter{Dynamic Analysis}
\label{cap:dynamic}

Following the static analysis, it became clear that a dynamic analysis would be
of the utmost importance. Understanding how the recommendation model responds to
users reinforcing its internal biases, like the ones already detected, could
potentially lead to a better understanding of how these systems favor certain
kinds of content.

In order for this analysis to be more true to reality, we implemented a simple
recommendation algorithm using TensorFlow Recommenders \citep{}, a library for
machine learning developed by Google for use with its TensorFlow \citep{}
framework. This means that, even though our model is deliberately bare-bones, it
conforms to industry-standard technology and practices.

The choice to use a simple recommendation algorithm instead of a more complex
one was twofold: first, we didn't want to use a model that could introduce many
confounding parameters to the analysis (e.g. hyperparameters, hardware
requirements, etc.), and second, we wanted to study a baseline that could, in
the future, be used as a comparison point for more complex algorithms.

The goal of this analysis is to gather data on how the recommendation system
behaves over time. As will be explained in the next sections, is to understand
what happens to the recommendation profile of the algorithm as it interacts with
itself via users that follow the generated suggestions.

The expectation is that the recommendation profile will grow ever more steep,
which is a reasonable guess; if the users reinforce the beliefs of the
algorithm, then is stands to reason that it will recommend popular movies with
more and more frequently, to more and more users. How much more frequently,
however, is the true question.

For the sake of clarity, let's imagine two users with very distinct preferences:
Alice, who enjoys adventure movies, and Bob, who enjoys horror movies. In
principle, the algorithm should have very different recommendations for both of
them and, were they to follow them, their custom suggestions should grow
increasingly different. At the end of this experiment, users like Alice would
all be recommended the same movies, and users like Bob would have their own set
of very popular films; we should expect, therefore, a multimodal distribution of
the recommendation frequencies, with "typical" adventure movies and "typical"
horror movies being much more popular than comedy, for example.

However, if the final recommendation profile looked like what was showcased in
the previous chapter, i.e. a very small subset of movies being recommended to
most users, then we could infer that the system devolved into a degenerate
feedback loop, ignoring personal preferences and distinctions between films.

\section{Datasets}
\label{sec:datasets04}

We used the same Movielens dataset as before and trained the recommender engine
with all available data. This is what we called the ``zeroth'' iteration. With the
resulting model, we generated a movie recommendation for each user in the
dataset and appended these into the original data as if users had actually
watched the movies we recommended. We repreated the training process, creating
what we called the ``first'' iteration, that is, the model that was generated
after \emph{one} round of recommendations.

This procedure was repeated two more times, resulting in a second and third
iterations. Since we generated one set recommendations with each model, this
means that, by the end, we had four sets, all of which were appended into the
one before, creating five total datasets: the original Movielens
(\verb|ratings0|), \verb|ratings0| with the recommendations from the zeroth
model appended (\verb|ratings1|), \verb|ratings1| with the recommendations from
the first model appended (\verb|ratings2|), and so on until \verb|ratings4|.

A series of analysis were conducted using these datasets as sources. We found
that, with each iteration, the recommendation profile got steeper and steeper,
that is, a few popular items got more recommended while the rest fell into
disfavor; this was predictably more noticeable in movies with higher average
ratings. In fact, a small set of around 20 movies were the only ones that
consistently rose in popularity with new iterations.

In order to understand how this ``feedback loop'' was developing, we fit
multiple regression models on our generated data. In the expression below,
\verb|pop| represents the popularity of a movie, \verb|t| represents the time
i.e. the iteration from 0 to 4, \verb|genre| represents the genre of a movie,
and \verb|movie_id| is the ID of a movie. Note that the genre was not used when
training the recommendation models described above, but it turned out that this
feature could explain a lot of the models' outputs.

\begin{verbatim}
  Family: nbinom2  ( log )
Formula:          pop ~ t * genre * rating + (1 | movie_id)
Data: features

     AIC      BIC   logLik deviance df.resid
 83863.8  84370.3 -41865.9  83731.8    15844

Random effects:

Conditional model:
 Groups   Name        Variance Std.Dev.
 movie_id (Intercept) 1.364    1.168
Number of obs: 15910, groups:  movie_id, 3182

Dispersion parameter for nbinom2 family ():   49

Conditional model:
                            Estimate Std. Error z value Pr(>|z|)
(Intercept)                -0.108714   0.272464  -0.399 0.689891
t                          -0.210272   0.021153  -9.941  < 2e-16 ***
genreAdventure              0.112129   0.574197   0.195 0.845174
genreAnimation             -2.286362   0.773873  -2.954 0.003132 **
genreChildren's             0.855196   0.755919   1.131 0.257915
genreComedy                -0.108159   0.352431  -0.307 0.758925
genreCrime                 -3.590470   0.850227  -4.223 2.41e-05 ***
genreDocumentary           -1.659325   1.508329  -1.100 0.271285
genreDrama                 -1.219969   0.409312  -2.981 0.002877 **
genreFilm-Noir            -16.633103   2.371128  -7.015 2.30e-12 ***
genreHorror                 0.512807   0.443025   1.158 0.247062
genreMusical               -4.259252   2.088483  -2.039 0.041410 *
genreMystery               -4.646920   1.461852  -3.179 0.001479 **
genreRomance               -2.118789   1.856289  -1.141 0.253699
genreSci-Fi                 0.363380   1.044431   0.348 0.727899
genreThriller               1.382908   0.855392   1.617 0.105944
genreWestern               -3.766758   2.108848  -1.786 0.074072 .
rating                      0.957533   0.085468  11.203  < 2e-16 ***
t:genreAdventure            0.161327   0.053403   3.021 0.002520 **
t:genreAnimation           -0.392012   0.067698  -5.791 7.01e-09 ***
t:genreChildren's           0.116028   0.066946   1.733 0.083068 .
t:genreComedy               0.119502   0.030504   3.918 8.94e-05 ***
t:genreCrime               -0.146821   0.085503  -1.717 0.085952 .
t:genreDocumentary          0.329357   0.195776   1.682 0.092507 .
t:genreDrama                0.005393   0.038739   0.139 0.889287
t:genreFilm-Noir           -0.730742   0.227275  -3.215 0.001303 **
t:genreHorror               0.148203   0.041813   3.544 0.000393 ***
t:genreMusical              0.265118   0.243060   1.091 0.275383
t:genreMystery             -1.150785   0.123034  -9.353  < 2e-16 ***
t:genreRomance              0.064921   0.227735   0.285 0.775590
t:genreSci-Fi              -0.311108   0.079779  -3.900 9.63e-05 ***
t:genreThriller             0.134758   0.077077   1.748 0.080403 .
t:genreWestern              0.184582   0.216581   0.852 0.394073
t:rating                    0.029594   0.006273   4.717 2.39e-06 ***
genreAdventure:rating      -0.209253   0.179840  -1.164 0.244606
genreAnimation:rating       0.594327   0.226301   2.626 0.008633 **
genreChildren's:rating     -0.473591   0.247951  -1.910 0.056131 .
genreComedy:rating         -0.194909   0.109223  -1.785 0.074341 .
genreCrime:rating           0.736375   0.243050   3.030 0.002448 **
genreDocumentary:rating    -0.113489   0.399299  -0.284 0.776242
genreDrama:rating          -0.031020   0.120957  -0.256 0.797601
genreFilm-Noir:rating       3.958891   0.595316   6.650 2.93e-11 ***
genreHorror:rating         -0.422452   0.151685  -2.785 0.005352 **
genreMusical:rating         0.944897   0.569196   1.660 0.096903 .
genreMystery:rating         1.092607   0.409047   2.671 0.007560 **
genreRomance:rating         0.013712   0.548422   0.025 0.980053
genreSci-Fi:rating         -0.423339   0.324094  -1.306 0.191477
genreThriller:rating       -0.729129   0.256016  -2.848 0.004400 **
genreWestern:rating         0.725673   0.578412   1.255 0.209625
t:genreAdventure:rating    -0.053250   0.015828  -3.364 0.000768 ***
t:genreAnimation:rating     0.110189   0.018603   5.923 3.16e-09 ***
t:genreChildren's:rating   -0.039499   0.020922  -1.888 0.059031 .
t:genreComedy:rating       -0.039790   0.008913  -4.464 8.04e-06 ***
t:genreCrime:rating         0.038460   0.022654   1.698 0.089557 .
t:genreDocumentary:rating  -0.088874   0.050610  -1.756 0.079076 .
t:genreDrama:rating        -0.006800   0.010754  -0.632 0.527202
t:genreFilm-Noir:rating     0.183567   0.054187   3.388 0.000705 ***
t:genreHorror:rating       -0.052874   0.013574  -3.895 9.81e-05 ***
t:genreMusical:rating      -0.082105   0.063538  -1.292 0.196285
t:genreMystery:rating       0.335177   0.032553  10.296  < 2e-16 ***
t:genreRomance:rating      -0.018634   0.064877  -0.287 0.773939
t:genreSci-Fi:rating        0.108210   0.023442   4.616 3.91e-06 ***
t:genreThriller:rating     -0.044310   0.022584  -1.962 0.049758 *
t:genreWestern:rating      -0.055461   0.056878  -0.975 0.329521
---
Signif. codes:  0 '***' 0.001 '**' 0.01 '*' 0.05 '.' 0.1 ' ' 1
\end{verbatim}

% Escrever o que eu fiz, sem preocupação com a formalidade. Mandar isso para o
% Patriota para ele lembrar do que a gente fez e marcar uma conversa para
% discutir se faz sentido e por onde a gente pode ir.
