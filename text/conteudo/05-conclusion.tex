%!TeX root=../tese.tex
%("dica" para o editor de texto: este arquivo é parte de um documento maior)
% para saber mais: https://tex.stackexchange.com/q/78101/183146

\chapter{Conclusion}
\label{cap:conclusion}

In this work, we studied recommender system bias. As detailed in
Chapter~\ref{cap:introduction}, these algorithms are pervasive in modern live
and have significant impact on our information diet; however, a growing body of
evidence indicates that these systems, specially when deployed in large social
networks, can display significant biases in their recommendations. Researchers
and activists alike worry that these bises might create echo chambers and
radicalization pipelines by amplifying extremist content. In broad terms, our
goal was to study one way through which these algorithms might be boosting
fringe viewpoints and radicalizing users: degenerate feedback loops.

Degeneracy in recommender systems has been explored before by
\citet{nguyen_exploring_2014} and \citet{jiang_degenerate_2019}. They concluded
that these algorithms show a tendency to reduce the diversity of recommended
content over time because of the feedback loop that occurs when the system must
learn from the its own outputs. Our main research objective was to further
characterize and model this behavior using both qualitative and quantitative
methods.

In Chapter~\ref{cap:review}, we reviewed the scientific literature that deals
with this topic and presented a few different viewpoints on the matter. Works by
\citet{hosseinmardi_evaluating_2020}, \citet{huszar_algorithmic_2021}, and
others suggest that social networks tend to amplify far-right views, while
\citet{munger_right-wing_2020} and \citet{ledwich_algorithmic_2019} posit that
this is not the case. \citet{ribeiro_auditing_2020} found evidence that, in
general, YouTube users tend to migrate from "lighter" content towards more
extreme videos, and \citet{stoica_algorithmic_2018} coined the term
``algorithmic glass ceiling'' to describe recommender system's propensity to
reinforce homophilic behavior.

We also drew attention to non-scientific endeavors that attempt to better
understand filter bubbles, echo chambers, and their impacts on the public. While
this issue is being debated in academia, journalists and whistleblowers like
\citet{ribeiro_como_2021} or \citet{wong_how_2021} hold social media companies
accountable by gathering anecdotal evidence and personal accounts about
algorithmic bias.

Our main contributions come in Chapters~\ref{cap:static} and \ref{cap:dynamic},
where we describe and conduct multiple experiments run on recommender
algorithms.

\section{Static Analysis}
\label{sec:static}

\section{Dynamic Analysis}
\label{sec:dynamic}
