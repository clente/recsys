%!TeX root=../tese.tex
%("dica" para o editor de texto: este arquivo é parte de um documento maior)
% para saber mais: https://tex.stackexchange.com/q/78101/183146

% O resumo é obrigatório, em português e inglês. Este comando também gera
% automaticamente a referência para o próprio documento, conforme as normas
% sugeridas da USP
\begin{resumo}{port}
Algoritmos de recomendação tornaram-se essenciais para o funcionamento de
diversos sistemas que usamos no dia a dia, desde quais filmes assistir até quais
produtos comprar. Entretanto, com a proliferação destes modelos nas redes
sociais, surgiram também novas preocupações. Evidências anedóticas e um corpo
cada vez mais robusto de pesquisa têm indicado que os algoritmos das redes
sociais, por valorizarem engajamento, podem estar radicalizando usuários através
da criação das chamadas câmaras de eco. Este trabalho pretende estudar
algoritmos de recomendação como sistemas dinâmicos de modo a identificar se seus
"espaços fásicos" estão sujeitos a dinâmicas de confinamento.
\end{resumo}

% O resumo é obrigatório, em português e inglês. Este comando também gera
% automaticamente a referência para o próprio documento, conforme as normas
% sugeridas da USP
\begin{resumo}{eng}
Recommendation algorithms have become essential to various day to day systems we
use, from what movies to watch to what products to buy. However, with the
proliferation of these models on social networks, new concerns have come to
light. Anecdotal evidence and an ever growing body of research indicate that
social network algorithms that promote engaging content might be radicalizing
users, creating what has become known as echo chambers. The present study aims
to study recommendation algorithms as dynamical systems as a means to identify
if their "phase spaces" are subject to confinement dynamics.
\end{resumo}
